\documentclass[11pt,]{article}
\usepackage[left=1in,top=1in,right=1in,bottom=1in]{geometry}
\newcommand*{\authorfont}{\fontfamily{phv}\selectfont}
\usepackage[]{mathpazo}


  \usepackage[T1]{fontenc}
  \usepackage[utf8]{inputenc}



\usepackage{abstract}
\renewcommand{\abstractname}{}    % clear the title
\renewcommand{\absnamepos}{empty} % originally center

\renewenvironment{abstract}
 {{%
    \setlength{\leftmargin}{0mm}
    \setlength{\rightmargin}{\leftmargin}%
  }%
  \relax}
 {\endlist}

\makeatletter
\def\@maketitle{%
  \newpage
%  \null
%  \vskip 2em%
%  \begin{center}%
  \let \footnote \thanks
    {\fontsize{18}{20}\selectfont\raggedright  \setlength{\parindent}{0pt} \@title \par}%
}
%\fi
\makeatother




\setcounter{secnumdepth}{3}


\usepackage{graphicx,grffile}
\makeatletter
\def\maxwidth{\ifdim\Gin@nat@width>\linewidth\linewidth\else\Gin@nat@width\fi}
\def\maxheight{\ifdim\Gin@nat@height>\textheight\textheight\else\Gin@nat@height\fi}
\makeatother
% Scale images if necessary, so that they will not overflow the page
% margins by default, and it is still possible to overwrite the defaults
% using explicit options in \includegraphics[width, height, ...]{}
\setkeys{Gin}{width=\maxwidth,height=\maxheight,keepaspectratio}

\title{Título\\
Subtítulo\\
Subtítulo  }



\author{\Large Darihana Linares Laureano\vspace{0.05in} \newline\normalsize\emph{Estudiante de Lic. en Geografía Mención Recursos Naturales y Ecoturismo,
Universidad Autónoma de Santo Domingo (UASD)}  }


\date{}

\usepackage{titlesec}

\titleformat*{\section}{\normalsize\bfseries}
\titleformat*{\subsection}{\normalsize\itshape}
\titleformat*{\subsubsection}{\normalsize\itshape}
\titleformat*{\paragraph}{\normalsize\itshape}
\titleformat*{\subparagraph}{\normalsize\itshape}

\titlespacing{\section}
{0pt}{36pt}{0pt}
\titlespacing{\subsection}
{0pt}{36pt}{0pt}
\titlespacing{\subsubsection}
{0pt}{36pt}{0pt}





\newtheorem{hypothesis}{Hypothesis}
\usepackage{setspace}

\makeatletter
\@ifpackageloaded{hyperref}{}{%
\ifxetex
  \PassOptionsToPackage{hyphens}{url}\usepackage[setpagesize=false, % page size defined by xetex
              unicode=false, % unicode breaks when used with xetex
              xetex]{hyperref}
\else
  \PassOptionsToPackage{hyphens}{url}\usepackage[unicode=true]{hyperref}
\fi
}

\@ifpackageloaded{color}{
    \PassOptionsToPackage{usenames,dvipsnames}{color}
}{%
    \usepackage[usenames,dvipsnames]{color}
}
\makeatother
\hypersetup{breaklinks=true,
            bookmarks=true,
            pdfauthor={Darihana Linares Laureano (Estudiante de Lic. en Geografía Mención Recursos Naturales y Ecoturismo,
Universidad Autónoma de Santo Domingo (UASD))},
             pdfkeywords = {palabra clave 1, palabra clave 2},  
            pdftitle={Título\\
Subtítulo\\
Subtítulo},
            colorlinks=true,
            citecolor=blue,
            urlcolor=blue,
            linkcolor=magenta,
            pdfborder={0 0 0}}
\urlstyle{same}  % don't use monospace font for urls

% set default figure placement to htbp
\makeatletter
\def\fps@figure{htbp}
\makeatother

\usepackage{pdflscape} \newcommand{\blandscape}{\begin{landscape}}
\newcommand{\elandscape}{\end{landscape}}


% add tightlist ----------
\providecommand{\tightlist}{%
\setlength{\itemsep}{0pt}\setlength{\parskip}{0pt}}

\begin{document}
	
% \pagenumbering{arabic}% resets `page` counter to 1 
%
% \maketitle

{% \usefont{T1}{pnc}{m}{n}
\setlength{\parindent}{0pt}
\thispagestyle{plain}
{\fontsize{18}{20}\selectfont\raggedright 
\maketitle  % title \par  

}

{
   \vskip 13.5pt\relax \normalsize\fontsize{11}{12} 
\textbf{\authorfont Darihana Linares Laureano} \hskip 15pt \emph{\small Estudiante de Lic. en Geografía Mención Recursos Naturales y Ecoturismo,
Universidad Autónoma de Santo Domingo (UASD)}   

}

}








\begin{abstract}

    \hbox{\vrule height .2pt width 39.14pc}

    \vskip 8.5pt % \small 

\noindent Resumen del manuscrito


\vskip 8.5pt \noindent \emph{Keywords}: palabra clave 1, palabra clave 2 \par

    \hbox{\vrule height .2pt width 39.14pc}



\end{abstract}


\vskip 6.5pt


\noindent  \section{Introducción}\label{introducciuxf3n}

La geomorfología como ciencia estudia las diferentes formas que
constituyen el relieve; estas formas muestran las distintas
características espaciales de una extensión, poniendo de manifiesto la
desigualdad en la estructura de lo que es el paisaje. Así como la forma
del relieve es el objeto de estudio de la Geomorfología, se entiende que
estas formas se producen por el efecto de una serie o series de acciones
sucesivas y regulares (Christofoletti (1988)). Por lo que la
Geomorfología es considerada la ciencia que estudia los procesos que
intervienen en la formación del relieve del relieve.

En geomorfología es posible distinguir la que estudia el relieve o
geomorfología estructural, y la que geomorfología que tiene su interés
en la forma del terreno considerando la acción erosiva o geomorfología
climática (Arroyo-González (2012)).

La geomorfología además de ser descriptiva debe ser cuantitativa, y así
dejar la dependencia de métodos pertenecientes a otros campos
científicos, como lo hace con la Física (Hubp (2010)).

\begin{figure}
\centering
\includegraphics{cuenca_de_Guayubin_1sFig.png}
\caption{Cuenca de Guayubin}
\end{figure}

\section{Metodología}\label{metodologuxeda}

\ldots

\section{Resultados}\label{resultados}

\ldots

\section{Discusión}\label{discusiuxf3n}

\section{Agradecimientos}\label{agradecimientos}

\section{Información de soporte}\label{informaciuxf3n-de-soporte}

\ldots

\section{\texorpdfstring{\emph{Script}
reproducible}{Script reproducible}}\label{script-reproducible}

\ldots

\section*{Referencias}\label{referencias}
\addcontentsline{toc}{section}{Referencias}

\hypertarget{refs}{}
\hypertarget{ref-arroyo2012esbozo}{}
Arroyo-González, L. N. (2012). Esbozo histórico de la geomorfología y su
papel como ciencia aplicada en el contexto de los peligros naturales y
los planes reguladores. \emph{Revista Geográfica de América Central},
\emph{1}(48), 15--34.

\hypertarget{ref-christofoletti1988geomorfologia}{}
Christofoletti, A. (1988). \emph{Geomorfologia}. Editora Blucher.

\hypertarget{ref-hubp2010gutierrez}{}
Hubp, J. L. (2010). \emph{Gutiérrez elorza, m.(2008), geomofología,
pearson/prentice hall, madrid, 898 p., isbn 97884832-23895}.




\newpage
\singlespacing 
\end{document}
