\documentclass[11pt,]{article}
\usepackage[left=1in,top=1in,right=1in,bottom=1in]{geometry}
\newcommand*{\authorfont}{\fontfamily{phv}\selectfont}
\usepackage[]{mathpazo}


  \usepackage[T1]{fontenc}
  \usepackage[utf8]{inputenc}



\usepackage{abstract}
\renewcommand{\abstractname}{}    % clear the title
\renewcommand{\absnamepos}{empty} % originally center

\renewenvironment{abstract}
 {{%
    \setlength{\leftmargin}{0mm}
    \setlength{\rightmargin}{\leftmargin}%
  }%
  \relax}
 {\endlist}

\makeatletter
\def\@maketitle{%
  \newpage
%  \null
%  \vskip 2em%
%  \begin{center}%
  \let \footnote \thanks
    {\fontsize{18}{20}\selectfont\raggedright  \setlength{\parindent}{0pt} \@title \par}%
}
%\fi
\makeatother




\setcounter{secnumdepth}{3}



\title{Título\\
Subtítulo\\
Subtítulo  }



\author{\Large Nombre del estudiante\vspace{0.05in} \newline\normalsize\emph{Afiliación, normalmente algo tal que ``Estudiante, Universidad Autónoma
de Santo Domingo (UASD)''}  }


\date{}

\usepackage{titlesec}

\titleformat*{\section}{\normalsize\bfseries}
\titleformat*{\subsection}{\normalsize\itshape}
\titleformat*{\subsubsection}{\normalsize\itshape}
\titleformat*{\paragraph}{\normalsize\itshape}
\titleformat*{\subparagraph}{\normalsize\itshape}

\titlespacing{\section}
{0pt}{36pt}{0pt}
\titlespacing{\subsection}
{0pt}{36pt}{0pt}
\titlespacing{\subsubsection}
{0pt}{36pt}{0pt}





\newtheorem{hypothesis}{Hypothesis}
\usepackage{setspace}

\makeatletter
\@ifpackageloaded{hyperref}{}{%
\ifxetex
  \PassOptionsToPackage{hyphens}{url}\usepackage[setpagesize=false, % page size defined by xetex
              unicode=false, % unicode breaks when used with xetex
              xetex]{hyperref}
\else
  \PassOptionsToPackage{hyphens}{url}\usepackage[unicode=true]{hyperref}
\fi
}

\@ifpackageloaded{color}{
    \PassOptionsToPackage{usenames,dvipsnames}{color}
}{%
    \usepackage[usenames,dvipsnames]{color}
}
\makeatother
\hypersetup{breaklinks=true,
            bookmarks=true,
            pdfauthor={Nombre del estudiante (Afiliación, normalmente algo tal que ``Estudiante, Universidad Autónoma
de Santo Domingo (UASD)'')},
             pdfkeywords = {palabra clave 1, palabra clave 2},  
            pdftitle={Título\\
Subtítulo\\
Subtítulo},
            colorlinks=true,
            citecolor=blue,
            urlcolor=blue,
            linkcolor=magenta,
            pdfborder={0 0 0}}
\urlstyle{same}  % don't use monospace font for urls

% set default figure placement to htbp
\makeatletter
\def\fps@figure{htbp}
\makeatother

\usepackage{pdflscape} \newcommand{\blandscape}{\begin{landscape}} \newcommand{\elandscape}{\end{landscape}}


% add tightlist ----------
\providecommand{\tightlist}{%
\setlength{\itemsep}{0pt}\setlength{\parskip}{0pt}}

\begin{document}
	
% \pagenumbering{arabic}% resets `page` counter to 1 
%
% \maketitle

{% \usefont{T1}{pnc}{m}{n}
\setlength{\parindent}{0pt}
\thispagestyle{plain}
{\fontsize{18}{20}\selectfont\raggedright 
\maketitle  % title \par  

}

{
   \vskip 13.5pt\relax \normalsize\fontsize{11}{12} 
\textbf{\authorfont Nombre del estudiante} \hskip 15pt \emph{\small Afiliación, normalmente algo tal que ``Estudiante, Universidad Autónoma
de Santo Domingo (UASD)''}   

}

}








\begin{abstract}

    \hbox{\vrule height .2pt width 39.14pc}

    \vskip 8.5pt % \small 

\noindent Resumen del manuscrito


\vskip 8.5pt \noindent \emph{Keywords}: palabra clave 1, palabra clave 2 \par

    \hbox{\vrule height .2pt width 39.14pc}



\end{abstract}


\vskip 6.5pt


\noindent  \hypertarget{introducciuxf3n}{%
\section{Introducción}\label{introducciuxf3n}}

\ldots

\hypertarget{metodologuxeda}{%
\section{Metodología}\label{metodologuxeda}}

\ldots

\hypertarget{resultados}{%
\section{Resultados}\label{resultados}}

\ldots

\hypertarget{discusiuxf3n}{%
\section{Discusión}\label{discusiuxf3n}}

\hypertarget{agradecimientos}{%
\section{Agradecimientos}\label{agradecimientos}}

\hypertarget{informaciuxf3n-de-soporte}{%
\section{Información de soporte}\label{informaciuxf3n-de-soporte}}

\ldots

\hypertarget{script-reproducible}{%
\section{\texorpdfstring{\emph{Script}
reproducible}{Script reproducible}}\label{script-reproducible}}

\ldots

\hypertarget{referencias}{%
\section{Referencias}\label{referencias}}




\newpage
\singlespacing 
\end{document}
