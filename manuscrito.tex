\documentclass[11pt,]{article}
\usepackage[left=1in,top=1in,right=1in,bottom=1in]{geometry}
\newcommand*{\authorfont}{\fontfamily{phv}\selectfont}
\usepackage[]{mathpazo}


  \usepackage[T1]{fontenc}
  \usepackage[utf8]{inputenc}



\usepackage{abstract}
\renewcommand{\abstractname}{}    % clear the title
\renewcommand{\absnamepos}{empty} % originally center

\renewenvironment{abstract}
 {{%
    \setlength{\leftmargin}{0mm}
    \setlength{\rightmargin}{\leftmargin}%
  }%
  \relax}
 {\endlist}

\makeatletter
\def\@maketitle{%
  \newpage
%  \null
%  \vskip 2em%
%  \begin{center}%
  \let \footnote \thanks
    {\fontsize{18}{20}\selectfont\raggedright  \setlength{\parindent}{0pt} \@title \par}%
}
%\fi
\makeatother




\setcounter{secnumdepth}{3}



\title{Título\\
Subtítulo\\
Subtítulo  }



\author{\Large Darihana Linares Laureano\vspace{0.05in} \newline\normalsize\emph{Estudiante de Lic. en Geografía Mención Recursos Naturales y Ecoturismo,
Universidad Autónoma de Santo Domingo (UASD)}  }


\date{}

\usepackage{titlesec}

\titleformat*{\section}{\normalsize\bfseries}
\titleformat*{\subsection}{\normalsize\itshape}
\titleformat*{\subsubsection}{\normalsize\itshape}
\titleformat*{\paragraph}{\normalsize\itshape}
\titleformat*{\subparagraph}{\normalsize\itshape}

\titlespacing{\section}
{0pt}{36pt}{0pt}
\titlespacing{\subsection}
{0pt}{36pt}{0pt}
\titlespacing{\subsubsection}
{0pt}{36pt}{0pt}





\newtheorem{hypothesis}{Hypothesis}
\usepackage{setspace}

\makeatletter
\@ifpackageloaded{hyperref}{}{%
\ifxetex
  \PassOptionsToPackage{hyphens}{url}\usepackage[setpagesize=false, % page size defined by xetex
              unicode=false, % unicode breaks when used with xetex
              xetex]{hyperref}
\else
  \PassOptionsToPackage{hyphens}{url}\usepackage[unicode=true]{hyperref}
\fi
}

\@ifpackageloaded{color}{
    \PassOptionsToPackage{usenames,dvipsnames}{color}
}{%
    \usepackage[usenames,dvipsnames]{color}
}
\makeatother
\hypersetup{breaklinks=true,
            bookmarks=true,
            pdfauthor={Darihana Linares Laureano (Estudiante de Lic. en Geografía Mención Recursos Naturales y Ecoturismo,
Universidad Autónoma de Santo Domingo (UASD))},
             pdfkeywords = {Geomorfología fluvial, Morfometria},  
            pdftitle={Título\\
Subtítulo\\
Subtítulo},
            colorlinks=true,
            citecolor=blue,
            urlcolor=blue,
            linkcolor=magenta,
            pdfborder={0 0 0}}
\urlstyle{same}  % don't use monospace font for urls

% set default figure placement to htbp
\makeatletter
\def\fps@figure{htbp}
\makeatother

\usepackage{pdflscape} \newcommand{\blandscape}{\begin{landscape}}
\newcommand{\elandscape}{\end{landscape}}


% add tightlist ----------
\providecommand{\tightlist}{%
\setlength{\itemsep}{0pt}\setlength{\parskip}{0pt}}

\begin{document}
	
% \pagenumbering{arabic}% resets `page` counter to 1 
%
% \maketitle

{% \usefont{T1}{pnc}{m}{n}
\setlength{\parindent}{0pt}
\thispagestyle{plain}
{\fontsize{18}{20}\selectfont\raggedright 
\maketitle  % title \par  

}

{
   \vskip 13.5pt\relax \normalsize\fontsize{11}{12} 
\textbf{\authorfont Darihana Linares Laureano} \hskip 15pt \emph{\small Estudiante de Lic. en Geografía Mención Recursos Naturales y Ecoturismo,
Universidad Autónoma de Santo Domingo (UASD)}   

}

}








\begin{abstract}

    \hbox{\vrule height .2pt width 39.14pc}

    \vskip 8.5pt % \small 

\noindent Resumen del manuscrito


\vskip 8.5pt \noindent \emph{Keywords}: Geomorfología fluvial, Morfometria \par

    \hbox{\vrule height .2pt width 39.14pc}



\end{abstract}


\vskip 6.5pt


\noindent  \section{Introducción}\label{introducciuxf3n}

Desde hace siglos atrás el hombre ha buscado la manera de explicar y
entender las distintas formas que el paisaje terrestre (relieve) posee.
Autores numerosos han investigado la génesis de estas nociones
geomorfológicas, remontándose a tres siglos atrás. Autores como Hutton,
Playfair y Lyell, sirvieron de antecesores o bases para la ciencia
geomorfológica. Tras su consolidación como ciencia en francia numerosos
autores fueron demostrando la importancia de esta ciencia, incluso
ramificándola (climática, eólica, litoral, glaciar, estructural,
tectónica, kárstica y fluvial; siendo la última de interés para esta
investigación), para mayor eficacia en sus estudios.

Los estudios en la geomorfología fluvial a nivel mundial son numerosos y
han servido para explicar cómo los drenajes de los ríos y sus redes
hidrográficas son importantes para la geomorfología, ya que estas redes
fluviales son parte de los procesos de modelado más activos en la
formación del relieve y que permiten mensurar la configuración del
mismo. Para los estudios en geomorfología fluvial, se hace uso del
análisis morfométrico de cuencas hidrográficas. La morfometría de cuenca
se ha convertido en la técnica cuantitativa para el estudio de las
cuencas de manera detallada y ordenada. Actualmente en la República
Dominicana el uso del análisis morfométrico para estudiar cuencas
hidrográficas es poco e insuficiente, pero no innecesario, a pesar de
que la República Dominicana goza de una diversa y extensa red de cuencas
hidrograficas, ricas y aprovechales para la aplicacion de diversas
tecnicas con el fin de explicar y enterder las propiedades del relieve y
su relación con las cuencas fluviales. Por lo que, este estudio es un
aporte para dar a conocer la configuracion y modelado de la cuenca
hidrografica del rio Guayubin, con el fin de fijar parámetros que
permitan evaluar esta cuenca fluvial; identificando el aspecto general
de la cuenca y de la red, el orden de red y análisis hortoniano, los
perfiles longitudinales e índice de concavidad de cursos más largos, y
la morfometría de cuenca. En ese mismo orden es imprescindible conocer
el concepto de cuenca fluvial o de drenaje. Conjunto de cuerpos de agua
con una área determinada que fluyen por distintos canales y escurren en
un mismo desague. Segun los autores Gregory y Walling, 1973; y Chorley,
1969 (como citó Gutiérrez Elorza (2008)), una cuenca fluvial compone el
espacio determinado en el que se suministran las aguas que discurren por
la superficie, el mismo está delimitado tanto por su relieve y su
hidrologia. Tambien considerada como una unidad imprescindible en
geomorfologica.

2.1 Revisión bibliográfica

Aspecto general de la cuenca y de la red El aspecto general de la cuenca
y de la red alude a la forma que adquiere la cuenca y a la forma de su
red de drenaje, según la confromación de sus ríos y el material rocoso
que la compone (patrones de drenaje). Varios autores conexión entre la
estructura que posee la red de drenaje con el material rocoso (Pedraza
Gilsanz (1996), Gutiérrez Elorza (2008), Howard (1967), Gregory \&
Walling (1973)).

2.3 Orden de red y análisis hortoniano

El orden de red hace referencia al orden en el que se clasifican los
cursos de agua, todo en base a su ramificación. Según Wikipedia (2020),
el orden de un curso de agua es siempre un número entero positivo que se
usa tanto en Geomorfología como en Hidrología para denotar la magnitud
de ramificación que posee una red fluvial. Para Bowden \& Wallis (1964),
el orden de red sostiene una relación entre las rocas con la
configuración de la red fluvial y con los procesos tanto hidrológicos
como erosivos. La clasificación de la red se hace de manera jerarquica.
Hoy dia existen múltiples normas para determinar la jerarquia de una
red: Strahler (1952), Horton (1945), Shreve (1967), Scheidegger (1970),
Leopold et al. (1964) Hack (1957) y Topological.

Perfiles longitudinales e índice de concavidad de cursos más largos

Morfometría de cuenca

preguntas de investigacion y explicacion de por que se hace el estudio
Aspecto general de la cuenca y de la red Orden de red y análisis
hortoniano Perfiles longitudinales e índice de concavidad de cursos más
largos Morfometría de cuenca

\section{Metodología}\label{metodologuxeda}

2.1 Área de estudio La cuenca del río Guayubín, abarca

\section{Resultados}\label{resultados}

\ldots

\section{Discusión}\label{discusiuxf3n}

\ldots

\section{Agradecimientos}\label{agradecimientos}

\ldots

\section{Información de soporte}\label{informaciuxf3n-de-soporte}

\ldots

\section{\texorpdfstring{\emph{Script}
reproducible}{Script reproducible}}\label{script-reproducible}

\ldots

\section*{Referencias}\label{referencias}
\addcontentsline{toc}{section}{Referencias}

\hypertarget{refs}{}
\hypertarget{ref-bowden1964effect}{}
Bowden, K. L., \& Wallis, J. R. (1964). Effect of stream-ordering
technique on horton's laws of drainage composition. \emph{Geological
Society of America Bulletin}, \emph{75}(8), 767--774.

\hypertarget{ref-gregory1973drainage}{}
Gregory, K. J., \& Walling, D. E. (1973). \emph{Drainage basin form and
process}.

\hypertarget{ref-gutierrez2008geomorfologia}{}
Gutiérrez Elorza, M. (2008). \emph{Geomorfología}.

\hypertarget{ref-howard1967drainage}{}
Howard, A. D. (1967). Drainage analysis in geologic interpretation: A
summation. \emph{AAPG Bulletin}, \emph{51}(11), 2246--2259.

\hypertarget{ref-pedraza1996geomorfologia}{}
Pedraza Gilsanz, J. de. (1996). \emph{Geomorfología: Principios, métodos
y aplicaciones}.

\hypertarget{ref-wikipedia2020stream}{}
Wikipedia, C. (2020). Stream order, tipo @ONLINE. Retrieved from
\url{https://en.wikipedia.org/wiki/Stream_order}




\newpage
\singlespacing 
\end{document}
